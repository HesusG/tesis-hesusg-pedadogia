%% resumen.tex
\chapter*{Resumen}
\addcontentsline{toc}{chapter}{Resumen}

La integración de la inteligencia artificial (IA) en los sistemas educativos es un reto que los gobiernos nacionales y los organismos internacionales abordan mediante políticas públicas heterogéneas. Esta investigación analiza 14 documentos de política sobre IA y educación de tres regiones (Europa, Américas y Asia-Pacífico) y dos organismos internacionales (UNESCO, Foro Económico Mundial), empleando un enfoque mixto que combina lectura comparativa cualitativa con análisis semántico computacional. Se desarrolló un pipeline basado en embeddings multilingües (paraphrase-multilingual-MiniLM-L12-v2) y ChromaDB que procesó 5{,}120 fragmentos de texto, generó una matriz de similitud coseno entre documentos y puntuó cada política en siete dimensiones: gobernanza, currículo, formación docente, infraestructura, ética, investigación y equidad.

Los resultados muestran que la orientación estratégica de una política predice mejor su contenido semántico que su ubicación geográfica. Se identificaron dos clusters: uno de estrategias tecnológicas (predominantemente asiáticas) y otro de estrategias integrales (transregional). Las políticas iberoamericanas (España, Brasil, Colombia) presentaron la mayor similitud del corpus (0.925--0.965), mientras que el EU AI Act se aisló como outlier por su naturaleza legislativa. La formación docente emergió como la dimensión con mayor brecha entre discurso e implementación concreta. A partir de los hallazgos se formulan orientaciones para una futura política mexicana de IA educativa.

\textbf{Palabras clave:} inteligencia artificial, educación, políticas públicas, análisis comparativo, procesamiento de lenguaje natural, análisis semántico, ChromaDB

\vspace{1cm}

\chapter*{Abstract}
\addcontentsline{toc}{chapter}{Abstract}

The integration of artificial intelligence (AI) into education systems is a challenge that governments and international organizations address through heterogeneous public policies. This study analyzes 14 AI education policy documents from three regions (Europe, the Americas, and Asia-Pacific) and two international organizations (UNESCO, World Economic Forum), using a mixed-methods approach that combines qualitative comparative reading with computational semantic analysis. A pipeline based on multilingual embeddings (paraphrase-multilingual-MiniLM-L12-v2) and ChromaDB processed 5,120 text fragments, generated a cosine similarity matrix between documents, and scored each policy across seven dimensions: governance, curriculum, teacher training, infrastructure, ethics, research, and equity.

Results show that a policy's strategic orientation better predicts its semantic content than its geographic location. Two clusters were identified: technology-focused strategies (predominantly Asian) and comprehensive strategies (cross-regional). Ibero-American policies (Spain, Brazil, Colombia) exhibited the highest similarity in the corpus (0.925--0.965), while the EU AI Act was isolated as an outlier due to its legislative nature. Teacher training emerged as the dimension with the greatest gap between policy discourse and concrete implementation. Based on these findings, guidelines are proposed for a future Mexican AI education policy.

\textbf{Keywords:} artificial intelligence, education, public policy, comparative analysis, natural language processing, semantic analysis, ChromaDB
