%% apendice-catalogo.tex — Catálogo de políticas analizadas
\chapter{Catálogo de Políticas Analizadas}
\label{ap:catalogo}

Este apéndice lista las 22 unidades de análisis del corpus, indicando cuáles fueron procesadas computacionalmente (\checkmark) y cuáles quedaron pendientes por restricciones de acceso (\texttimes).

\section*{Europa}

\begin{description}
\item[\checkmark\ Unión Europea: EU AI Act (2024)] Regulation (EU) 2024/1689 del Parlamento Europeo y del Consejo. Regulación sobre inteligencia artificial por niveles de riesgo. 620{,}000 caracteres. Idioma: inglés. \citep{eu2024aiact}

\item[\checkmark\ España: ENIA (2020)] Estrategia Nacional de Inteligencia Artificial. Plan estratégico integral con ejes de gobernanza, talento, infraestructura y ética. 170{,}000 caracteres. Idioma: español. \citep{spain2020enia}

\item[\checkmark\ Francia: Villani Report (2018)] \textit{For a Meaningful AI: Towards a French and European Strategy.} Informe del matemático Cédric Villani con recomendaciones en investigación, educación, ética y empleo. 424{,}000 caracteres. Idioma: inglés (traducción oficial). \citep{villani2018ai}

\item[\texttimes\ Alemania: KI-Strategie (2018)] Estrategia nacional de IA del gobierno federal. Archivo descargado como HTML en lugar de PDF; requiere descarga manual. \citep{germany2018aistrategy}

\item[\texttimes\ Finlandia: Finland's Age of AI (2017)] Estrategia temprana de IA. Documento no disponible en acceso abierto directo.

\item[\texttimes\ Estonia: KrattAI (2019)] Informe del grupo de trabajo estonio sobre IA. Publicado en el sitio del gobierno estonio; requiere descarga manual.
\end{description}

\section*{Américas}

\begin{description}
\item[\checkmark\ Canadá: Pan-Canadian AI Strategy (2017)] Primera estrategia nacional de IA del mundo. Enfoque en investigación y formación de talento. 6{,}400 caracteres (documento breve). Idioma: inglés. \citep{canada2017ai}

\item[\texttimes\ Estados Unidos: Executive Order 14110 (2023)] Orden ejecutiva sobre IA segura. Publicada en el Federal Register; requiere descarga manual.

\item[\texttimes\ México] No cuenta con una estrategia nacional de IA vigente. Existen antecedentes de C Minds (2018) y documentos sectoriales de la SEP. \citep{cminds2018iamexico}

\item[\checkmark\ Brasil: EBIA (2021)] Estratégia Brasileira de Inteligência Artificial. Plan con ejes de legislación, gobernanza, investigación, educación y seguridad. 132{,}000 caracteres. Idioma: portugués. \citep{brasil2021ebia}

\item[\texttimes\ Chile: Política Nacional de IA (2021)] Publicada por el Ministerio de Ciencia; requiere descarga desde el sitio ministerial.

\item[\checkmark\ Colombia: CONPES 3975 (2019)] Política Nacional para la Transformación Digital e Inteligencia Artificial. Documento del Consejo Nacional de Política Económica y Social. 141{,}000 caracteres. Idioma: español. \citep{colombia2019conpes}
\end{description}

\section*{Asia-Pacífico}

\begin{description}
\item[\texttimes\ China: NGAIDP (2017)] Plan de Desarrollo de Inteligencia Artificial de Nueva Generación. Disponible en traducción al inglés de DigiChina (Stanford); requiere descarga manual.

\item[\checkmark\ Japón: AI Strategy 2019 (2019)] \textit{AI for Everyone: People, Industries, Regions and Governments.} Estrategia articulada en torno al concepto de ``Society 5.0''. 129{,}000 caracteres. Idioma: inglés. \citep{japan2019ai}

\item[\checkmark\ Corea del Sur: National AI Strategy (2019)] Estrategia nacional con enfoque en semiconductores, datos y talento en IA. 130{,}000 caracteres. Idioma: inglés (traducción OECD.AI). \citep{korea2019ai}

\item[\checkmark\ Singapur: NAIS (2019)] National AI Strategy. Cinco proyectos nacionales de IA y ecosistema de investigación. 100{,}000 caracteres. Idioma: inglés. \citep{singapore2019nais}

\item[\checkmark\ India: \#AIForAll (2018)] Estrategia del NITI Aayog. Aplicaciones sectoriales de IA con énfasis en inclusión. 274{,}000 caracteres. Idioma: inglés. \citep{india2018niti}

\item[\checkmark\ India: NEP 2020 (2020)] National Education Policy. Política educativa integral con componentes de tecnología y pensamiento computacional. 273{,}000 caracteres. Idioma: inglés. \citep{india2020nep}

\item[\checkmark\ Australia: AI Action Plan (2021)] Plan de acción con inversión en investigación, adopción industrial y regulación. 72{,}000 caracteres. Idioma: inglés. \citep{australia2021aiaction}
\end{description}

\section*{Organismos Internacionales}

\begin{description}
\item[\checkmark\ UNESCO: Guidance for GenAI (2023)] Orientaciones para la IA generativa en educación e investigación. 165{,}000 caracteres. Idioma: inglés. \citep{unesco2023genai}

\item[\texttimes\ OCDE] Recomendación del Consejo sobre IA (2019) y Digital Education Outlook (2021). Documentos con acceso restringido o de pago.

\item[\checkmark\ Foro Económico Mundial: Future of Jobs (2020)] Informe sobre el futuro del empleo y las competencias necesarias ante la automatización y la IA. 430{,}000 caracteres. Idioma: inglés. \citep{wef2020future}

\item[\texttimes\ Banco Mundial] \textit{Reimagining Human Connections} (2020). Disponible en el repositorio abierto del Banco Mundial; requiere búsqueda manual.
\end{description}
