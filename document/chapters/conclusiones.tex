%% conclusiones.tex — Conclusiones
\chapter{Conclusiones}
\label{cap:conclusiones}

% Target: ~5 páginas (~1,250 palabras)
% Iteration 1 — 2026-02-23

Esta tesis analizó 14 políticas públicas sobre inteligencia artificial de tres regiones geográficas y dos organismos internacionales mediante un enfoque mixto que combinó lectura comparativa cualitativa con análisis semántico computacional. Las conclusiones se organizan en respuesta a los objetivos específicos planteados en la Sección~\ref{subsec:objetivos-especificos}.

\section{Respuesta a los Objetivos de Investigación}

\textbf{OE1: Construir un corpus documental sistematizado de políticas públicas sobre educación en IA.} El corpus resultante comprende 22 unidades de análisis (5 países europeos y la UE, 6 americanos, 6 de Asia-Pacífico y 4 organismos internacionales), de las cuales 14 fueron procesadas computacionalmente. Las políticas abarcan el período 2017--2024. La categorización por género discursivo reveló tres tipos: estrategias nacionales integrales, políticas sectoriales y regulación legislativa.

\textbf{OE2: Desarrollar un marco analítico de siete dimensiones para la comparación estructurada de las políticas.} El marco de siete dimensiones permitió comparar las políticas de manera sistemática. Las mayores convergencias entre regiones se dieron en ética y valores; la mayor variación en puntuaciones se observó en infraestructura; y la mayor brecha entre intención declarativa y concreción operativa apareció en formación docente. Asia-Pacífico lideró en investigación e innovación y currículo; Iberoamérica en equidad e inclusión.

\textbf{OE3: Implementar una herramienta de análisis semántico basada en embeddings y ChromaDB.} El pipeline procesó 5{,}120 fragmentos de texto mediante embeddings multilingües y similitud coseno, generando una matriz de $14 \times 14$. Los pares de mayor similitud fueron España--Brasil (0.965) y Japón--Corea del Sur (0.930). El clustering jerárquico identificó dos agrupaciones temáticas y un outlier regulatorio (EU AI Act).

\textbf{OE4: Identificar patrones de similitud y divergencia entre las políticas analizadas.} Tres patrones principales: (a) convergencia iberoamericana transatlántica (España, Brasil, Colombia comparten estructura y prioridades); (b) cluster tecnológico asiático (Japón, Corea, Singapur, India NITI, más Canadá y Australia); (c) la orientación estratégica predice mejor el contenido semántico de una política que su ubicación geográfica.

\textbf{OE5: Formular recomendaciones para la política educativa de IA en México.} A partir del análisis comparativo se derivaron cuatro orientaciones: (a) las estrategias iberoamericanas (España, Brasil, Colombia) como referentes; (b) la formación docente como dimensión prioritaria; (c) un modelo de gobernanza mixto adaptado al contexto institucional de la SEP; y (d) la infraestructura como precondición, a la luz de la brecha digital rural.


\section{Hallazgos Principales}

Tres hallazgos merecen énfasis por su contribución a la educación comparada:

\begin{enumerate}
\item \textbf{Las fronteras regionales no determinan el contenido de las políticas.} La presencia de Francia junto con Brasil y Colombia en el mismo cluster, o de Canadá junto con políticas asiáticas, muestra que la clasificación geográfica, dominante en la literatura comparativa \citep{bray1995levels}, no captura la variación real en el contenido de las políticas de IA educativa.

\item \textbf{El análisis semántico computacional complementa, no sustituye, la lectura comparativa.} La triangulación mostró convergencia en cinco de siete dimensiones pero también reveló limitaciones: los embeddings capturan proximidad temática del vocabulario sin distinguir entre intención declarativa y concreción operativa. Esta limitación delimita el alcance de la herramienta como generadora de hipótesis, no como sustituto del juicio cualitativo.

\item \textbf{La formación docente es la dimensión más desatendida del corpus.} Diez de 14 políticas la mencionan, pero solo tres proponen mecanismos concretos. Este desfase entre retórica y operacionalización constituye la brecha más consistente del análisis y sugiere un área prioritaria para las políticas en desarrollo, incluida la de México.
\end{enumerate}


\section{Limitaciones}

\begin{itemize}
\item \textbf{Corpus incompleto.} De las 22 unidades de análisis definidas, 8 no fueron procesadas por restricciones de acceso a los documentos fuente. La ausencia de Estados Unidos, China, Alemania y la OCDE limita la representatividad del análisis, particularmente en las dimensiones de gobernanza e investigación donde estos actores son centrales.

\item \textbf{Modelo de embeddings.} El modelo local (\texttt{paraphrase-multilingual-MiniLM-L12-v2}, 384 dimensiones) fue elegido por su reproducibilidad, pero modelos de mayor dimensionalidad podrían capturar matices semánticos adicionales. La extensión desigual de los documentos (6{,}000 a 620{,}000 caracteres) influye en la representación promedio de cada política.

\item \textbf{Análisis estático.} El corpus captura un momento temporal (2017--2024) sin rastrear la evolución de las políticas. Varias naciones han actualizado sus estrategias desde su publicación original.

\item \textbf{Idioma.} Aunque el modelo es multilingüe, la proximidad entre español y portugués puede inflar la similitud del bloque iberoamericano. No fue posible aislar el componente lingüístico del componente temático.
\end{itemize}


\section{Trabajo Futuro}

\begin{enumerate}
\item \textbf{Completar el corpus.} Incorporar los 8 documentos pendientes , particularmente Estados Unidos (EO 14110), China (NGAIDP), Alemania y la OCDE, para alcanzar la cobertura planeada de 22 unidades.

\item \textbf{Análisis longitudinal.} Rastrear las actualizaciones de políticas (Japón 2022, Corea 2023, Singapur 2.0) para evaluar convergencia o divergencia temporal.

\item \textbf{Validación con modelos alternativos.} Replicar el análisis con \texttt{text-embedding-3-small} (OpenAI, 1{,}536 dimensiones) para comparar resultados y evaluar la robustez de los patrones identificados.

\item \textbf{Estudio de implementación.} La convergencia discursiva identificada no implica convergencia en la implementación. Un estudio de seguimiento que examine la traducción de estas políticas en acciones concretas (presupuestos, programas, evaluaciones) completaría el análisis documental.

\item \textbf{Propuesta para México.} Desarrollar, a partir de los insumos comparativos de esta tesis, un documento de política de IA educativa adaptado al contexto institucional mexicano, con atención particular a la formación docente y la infraestructura.
\end{enumerate}
