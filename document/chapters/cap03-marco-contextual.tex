%% cap03-marco-contextual.tex — Marco Contextual
\chapter{Marco Contextual}
\label{cap:marco-contextual}

% Target: ~20 páginas (~5,000 palabras)
% Iteration 2 — reviewed (writing/weakpoints/proofread), fixes applied

Este capítulo presenta el panorama de las políticas públicas sobre educación en inteligencia artificial en las 22 unidades de análisis que componen el corpus de esta investigación. La exposición se organiza por regiones (Europa, Américas y Asia-Pacífico) y concluye con una síntesis comparativa preliminar que anticipa los ejes del análisis formal desarrollado en el capítulo de resultados.


\section{Panorama Global}
\label{sec:panorama-global}

Para mediados de 2024, más de 60 países contaban con alguna forma de estrategia nacional de inteligencia artificial~\citep{maslej2024aiindex}. Sin embargo, la mayoría de estas estrategias se orientan hacia la competitividad económica, la investigación y la industria; la educación ocupa un papel secundario o periférico en la mayor parte de los documentos. \citet{fatima2020aipolicy} analizaron las estrategias nacionales de IA de más de 30 países y encontraron que, si bien aproximadamente el 90\% mencionaban la formación de talento y el desarrollo de la fuerza laboral, las provisiones educativas específicas variaban ampliamente en profundidad y alcance.

Los cuatro organismos internacionales incluidos en este estudio han producido marcos de referencia que influyen en las políticas nacionales. La UNESCO ha publicado el Consenso de Beijing sobre IA y Educación~\citep{unesco2019beijing}, la Guía de IA y Educación para Formuladores de Políticas~\citep{miao2021guidance} y la Guía sobre IA Generativa en Educación e Investigación~\citep{unesco2023genai}; estos tres documentos constituyen el marco normativo más completo para la integración de IA en los sistemas educativos. La OCDE ha contribuido con sus Principios de IA~\citep{oecd2019ai} y el \textit{Digital Education Outlook}~\citep{oecd2021digitaloutlook}, además de mantener el Observatorio de Políticas de IA con más de 800 iniciativas registradas. El Foro Económico Mundial, a través de su red de Centros para la Cuarta Revolución Industrial y sus informes sobre el futuro del empleo, ha incidido en la agenda de habilidades y IA en la educación. El Banco Mundial ha integrado la IA en sus programas de asistencia técnica para sistemas educativos en países en desarrollo, aunque sus contribuciones específicas al marco normativo de IA en educación son menos explícitas que las de la UNESCO y la OCDE.

\citet{jobin2019landscape} identificaron convergencia en torno a cinco principios éticos en 84 directrices de IA (transparencia, justicia, no maleficencia, responsabilidad y privacidad), pero señalaron que la traducción de estos principios a políticas educativas concretas permanecía pendiente.

La respuesta de los países puede agruparse en tres categorías. Un primer grupo adoptó estrategias tempranas con componentes educativos explícitos: China (2017), Canadá (2017), Finlandia (2017), Francia (2018), Singapur (2019) y Corea del Sur (2019, con plan educativo específico en 2020). Un segundo grupo formuló estrategias nacionales de IA con menciones educativas de menor profundidad: Alemania (2018), Japón (2019), India (2018), Estonia (2019), España (2020), Chile (2021), Colombia (2019), Brasil (2021) y Australia (2021). Un tercer grupo carecía de estrategia formal o tenía esfuerzos fragmentados: Estados Unidos y México~\citep{unesco2023genai}. La Unión Europea ocupa una posición particular como regulador supranacional con el AI Act (2024), que clasifica la educación como sector de alto riesgo.


\section{Europa}
\label{sec:europa}

\subsection{Unión Europea}

La Unión Europea adoptó el Reglamento 2024/1689, conocido como AI Act, en junio de 2024, el primer marco legal integral sobre IA en el mundo~\citep{eu2024aiact}. El reglamento clasifica los sistemas de IA por nivel de riesgo. La educación se identifica como sector de ``alto riesgo'' en el Anexo III: los sistemas utilizados para determinar el acceso a instituciones educativas, evaluar resultados de aprendizaje o monitorear estudiantes durante exámenes requieren evaluaciones de conformidad, transparencia, supervisión humana y documentación.

En paralelo, el Plan de Acción de Educación Digital 2021--2027~\citep{eu2020deap} establece dos prioridades estratégicas: fomentar un ecosistema de educación digital de alto rendimiento y mejorar las competencias digitales. La Acción 6 produjo directrices éticas sobre el uso de IA y datos en la enseñanza para educadores, publicadas en 2022. El marco DigComp 2.2 actualizó las competencias digitales europeas para incluir 82 ejemplos de conocimientos, habilidades y actitudes relacionados con la IA. La Comisión destinó aproximadamente 2,100 millones de euros del Programa Europa Digital (2021--2027) a IA. No obstante, la implementación curricular depende de cada Estado miembro, lo que genera una adopción desigual entre los 27 países.

\subsection{España}

España publicó la Estrategia Nacional de Inteligencia Artificial (ENIA) en diciembre de 2020, con una inversión proyectada de 600 millones de euros para 2021--2023~\citep{spain2020enia}. El Eje 2 de la ENIA aborda el talento y las habilidades en IA: aumentar los programas universitarios, fortalecer las competencias digitales en educación básica y promover la recualificación profesional. El Plan Nacional de Competencias Digitales (2021) complementa la ENIA con medidas para integrar el pensamiento computacional en la educación formal.

El Instituto Nacional de Tecnologías Educativas y de Formación del Profesorado (INTEF) ha desarrollado el Marco de Referencia de la Competencia Digital Docente, actualizado en 2022 para alinearse con el DigCompEdu europeo. La Ley Orgánica de Educación (LOMLOE, 2020) introdujo el pensamiento computacional como elemento transversal, aunque no menciona explícitamente la IA. Las Comunidades Autónomas retienen la competencia sobre la implementación curricular, lo que genera variación regional considerable.

\subsection{Francia}

Francia lanzó su estrategia nacional de IA tras el Informe Villani de marzo de 2018, con un compromiso inicial de 1,500 millones de euros para 2018--2022~\citep{villani2018ai}. La segunda fase (2021) añadió 2,200 millones de euros hasta 2025 con énfasis en formación y talento. Se establecieron cuatro institutos interdisciplinarios de IA (3IA) en Grenoble, Niza, París y Toulouse como centros de investigación y educación.

En educación básica, Francia es notable por el Partenariat d'Innovation en Intelligence Artificielle (P2IA), lanzado en 2019 por la Direction du Numérique pour l'Éducation. El P2IA desarrolló herramientas de aprendizaje adaptativo basadas en IA para lectura y matemáticas en primaria (Lalilo, Adaptiv'Math, Navi) desplegadas en miles de aulas, un despliegue de tutores de IA a escala en escuelas públicas que pocos países han replicado. En secundaria, la asignatura Numérique et Sciences Informatiques (NSI), disponible desde 2019, incluye módulos de IA. La asignatura obligatoria Sciences Numériques et Technologie (SNT) para todos los alumnos de Seconde introduce conceptos de datos, algoritmos e IA.

\subsection{Alemania}

Alemania adoptó su Estrategia Nacional de IA en noviembre de 2018, actualizada en 2020 con un incremento del compromiso federal de 3,000 a 5,000 millones de euros hasta 2025~\citep{germany2018aistrategy, germany2020aiupdate}. La estrategia aborda la educación bajo el pilar ``Sociedad'', con medidas para integrar la IA en todos los niveles educativos y la creación de 100 cátedras adicionales de IA en universidades.

El KI-Campus (Campus de IA), financiado por el Ministerio Federal de Educación e Investigación (BMBF) desde 2019, ofrece más de 30 cursos abiertos sobre IA dirigidos a estudiantes universitarios, profesionales y educadores. La Plattform Lernende Systeme, establecida en 2017, funciona como órgano asesor con un grupo de trabajo sobre IA en educación. Sin embargo, la estructura federal de Alemania , con 16 \textit{Länder} soberanos en materia educativa, genera fragmentación: las decisiones curriculares recaen en los ministerios estatales, coordinados a través de la Kultusministerkonferenz (KMK). La estrategia ``Bildung in der digitalen Welt'' (2016, suplementada en 2021) aborda competencias digitales pero no incluye IA de forma integral en la educación K-12.

\subsection{Finlandia}

Finlandia destaca por el curso \textit{Elements of AI}, desarrollado en 2018 por la Universidad de Helsinki y la empresa Reaktor~\citep{tuomi2018impact}. Diseñado con la meta de que el 1\% de la población finlandesa adquiriera conocimientos básicos de IA, el curso superó este objetivo en meses, ha sido traducido a más de 25 idiomas y acumula más de un millón de participantes registrados globalmente. Finlandia lo ofreció como obsequio a la UE durante su presidencia del Consejo en 2019.

El programa AuroraAI (2020--2022), con un presupuesto de 4.6 millones de euros, desarrolló una red de servicios públicos basada en IA, incluyendo trayectorias educativas. El Currículo Nacional Básico de 2014 (implementado desde 2016) introdujo competencias transversales en TIC y pensamiento computacional desde primer grado, con un enfoque de aprendizaje basado en fenómenos que facilita la integración interdisciplinaria de temas tecnológicos. El enfoque finlandés privilegia la alfabetización ciudadana en IA por encima de la formación de especialistas, mediante un modelo de colaboración público-privada. Una limitación reconocida es la escasez de evaluaciones sistemáticas de los resultados de alfabetización en IA en educación básica~\citep{oecd2021digitaloutlook}.

\subsection{Estonia}

Estonia publicó su estrategia de IA, conocida como KrattAI, en julio de 2019, con una asignación inicial de aproximadamente 10 millones de euros para 2019--2021~\citep{estonia2019krattai}. Más que un programa de inversión masiva, KrattAI se centró en crear condiciones favorables para la adopción de IA, identificando más de 50 casos de uso en servicios gubernamentales, incluida la educación.

El ecosistema digital educativo de Estonia se apoya en décadas de inversión. El programa ProgeTiiger, lanzado en 2012, introdujo la programación y el pensamiento computacional desde primer grado; para 2020, más del 85\% de las escuelas de educación general habían participado. La infraestructura digital del país (X-Road, e-governance, las plataformas eKool y Stuudium) proporciona un entorno propicio para herramientas educativas basadas en IA. La Estrategia Educativa 2021--2035 enfatiza competencias digitales, trayectorias de aprendizaje personalizadas y analítica de datos en educación. La agilidad de Estonia como Estado pequeño (1.3 millones de habitantes) permite pilotar programas a escala nacional con rapidez, aunque su presupuesto es modesto frente a economías mayores y el desarrollo de recursos educativos en estonio presenta desafíos propios de las lenguas minoritarias.


\section{Américas}
\label{sec:americas}

\subsection{Estados Unidos}

Estados Unidos carece de una política federal unificada de IA en educación. El panorama regulatorio consiste en acciones ejecutivas superpuestas, propuestas legislativas y marcos voluntarios. La Orden Ejecutiva 14110 (octubre de 2023) sobre desarrollo seguro de IA instruyó al Departamento de Educación a desarrollar recursos sobre IA en la enseñanza y a la NSF a financiar investigación en educación e IA~\citep{whitehouse2023aeo}. El \textit{Blueprint for an AI Bill of Rights} (2022) del OSTP identificó cinco principios aplicables a contextos educativos pero sin carácter vinculante. En mayo de 2023, la Oficina de Tecnología Educativa del Departamento de Educación publicó un informe de 71 páginas sobre IA y el futuro de la enseñanza, como guía no regulatoria.

Ante la ausencia de mandatos federales, las políticas han emergido en los estados , varios de los cuales habían legislado sobre IA en educación para principios de 2025, y mediante iniciativas no gubernamentales. El marco AI4K12 (2018), desarrollado por la AAAI y la CSTA, propuso las ``Cinco Grandes Ideas de la IA'' para guiar el currículo K-12, pero su adopción es voluntaria~\citep{maslej2024aiindex}. La NSF invirtió aproximadamente 140 millones de dólares en institutos nacionales de investigación en IA, varios de ellos enfocados en educación. Un cambio de administración en 2025 rescindió porciones de la Orden Ejecutiva 14110, fragmentando aún más el panorama.

\subsection{Canadá}

Canadá fue uno de los primeros países en adoptar una estrategia nacional de IA. La Pan-Canadian AI Strategy se lanzó en 2017 con una inversión inicial de 125 millones de dólares canadienses, administrada por el Canadian Institute for Advanced Research (CIFAR)~\citep{canada2017ai}. La estrategia financió tres institutos nacionales: el Vector Institute (Toronto), Mila (Montreal) y Amii (Edmonton). En 2021, la estrategia fue renovada con 443.8 millones adicionales, elevando el compromiso total a aproximadamente 568.8 millones de dólares canadienses.

La estrategia se enfoca en investigación de posgrado y desarrollo de talento, no en educación K-12, que en Canadá es competencia provincial. Esta división de poderes genera fragmentación: British Columbia, Ontario, Quebec y Alberta han introducido componentes de computación y alfabetización digital en sus currículos provinciales, pero el contenido específico de IA varía. No existe un marco nacional de alfabetización en IA para K-12 ni un estándar de formación docente en IA. La fortaleza de Canadá reside en su ecosistema de investigación, con universidades entre las mejores del mundo en IA, pero esta fortaleza no se ha traducido en integración curricular sistemática en la educación preuniversitaria.

\subsection{México}

México no cuenta con una estrategia nacional de IA formalmente adoptada. El esfuerzo más significativo fue el proceso IA2030MX, iniciado en 2018 por la coalición C~Minds en colaboración con Oxford Insights~\citep{cminds2018iamexico}. El documento resultante, \textit{Hacia una Estrategia de IA en México}, propuso recomendaciones en gobernanza, investigación, fuerza laboral, infraestructura y ética, e incluyó sugerencias sobre integración curricular y formación docente. La transición de gobierno de 2018 interrumpió el seguimiento institucional; la administración entrante no adoptó el documento como política oficial.

En el sector educativo, las brechas son sustanciales. El Programa Sectorial de Educación 2020--2024~\citep{sep2020sectorial} omite la IA. La Nueva Escuela Mexicana~\citep{sep2022nem} organiza los contenidos en campos formativos con enfoque comunitario y humanístico, sin abordar la IA ni el pensamiento computacional. El programa \textit{@prende 2.0} fue desfinanciado a partir de 2019. El CONAHCYT reorientó sus prioridades hacia las ``ciencias humanísticas'', reduciendo el financiamiento para investigación tecnológica. México ocupa el lugar 55 en el Índice de Preparación para la IA de Oxford Insights y el cuarto en América Latina, detrás de Chile, Brasil y Colombia~\citep{oxfordinsights2023}. La ENDUTIH 2023 reporta que, pese a que el 79.5\% de la población usa internet, la conectividad en zonas rurales (62.3\%) y en escuelas públicas es considerablemente menor~\citep{inegi2024endutih}.

\subsection{Brasil}

Brasil publicó su Estrategia Brasileña de Inteligencia Artificial (EBIA) en abril de 2021, coordinada por el Ministerio de Ciencia, Tecnología e Innovación~\citep{brasil2021ebia}. La EBIA se organiza en nueve ejes estratégicos, de los cuales la educación es el primero, una elección simbólica que distingue al documento de otras estrategias nacionales donde la educación es periférica. El eje educativo propone promover la alfabetización en IA en todos los niveles, integrar contenidos de IA en la Base Nacional Comum Curricular (BNCC), formar docentes y utilizar herramientas de IA para reducir desigualdades educativas.

Sin embargo, la BNCC (actualizada por última vez en 2018) incluye competencias digitales como una de diez competencias generales pero no referencia específicamente la IA, dado que fue elaborada antes de la EBIA. El Programa de Innovación Educación Conectada (PIEC, 2017) destinó aproximadamente 3,100 millones de reales a conectividad escolar, aunque la implementación ha sido desigual. Una brecha notable es la distancia entre las ambiciones del eje educativo de la EBIA y su implementación: la estrategia carece de asignaciones presupuestarias específicas, cronogramas o metas medibles para sus objetivos educativos.

\subsection{Chile}

Chile publicó su Política Nacional de Inteligencia Artificial en noviembre de 2021, producida por el Ministerio de Ciencia, Tecnología, Conocimiento e Innovación~\citep{chile2021ia}. Fue el primer país latinoamericano en adoptar una política integral de IA. La educación figura como componente del habilitador ``talento y capital humano'', con acciones propuestas para integrar el pensamiento computacional y conceptos de IA en el currículo K-12, crear trayectorias de IA en la formación docente y expandir las habilidades digitales mediante el SENCE. La política fue desarrollada con la participación de más de 1,200 personas en consultas públicas.

El Plan Nacional de Lenguajes Digitales (2019), operado por el Ministerio de Educación y la Fundación Kodea, introdujo el pensamiento computacional y la programación en los grados 5 a 12. Para 2022, el programa había alcanzado aproximadamente 2,000 escuelas y capacitado a más de 4,000 docentes, aunque aún no cubre contenidos específicos de IA. Chile ocupa el primer lugar en América Latina y el 39 en el Índice de Oxford Insights~\citep{oxfordinsights2023}. Sin embargo, la integración de IA en el currículo nacional permanece en fase piloto y la formación docente en IA no ha sido escalada más allá de proyectos demostrativos.

\subsection{Colombia}

Colombia adoptó su política de IA mediante el CONPES 3975, aprobado en noviembre de 2019~\citep{colombia2019conpes}. El mecanismo CONPES otorga al documento carácter de directiva vinculante para el poder ejecutivo. El pilar de capital humano incluye compromisos de fortalecer la educación STEM, incorporar el pensamiento computacional y fundamentos de IA en el currículo, formar docentes en tecnologías digitales y expandir programas de habilidades digitales. La inversión estimada fue de aproximadamente 21,500 millones de pesos colombianos (cerca de 5.8 millones de dólares), cifra modesta frente a la ambición de los objetivos.

La Misión Internacional de Sabios (2019) formuló recomendaciones específicas sobre IA: integrar el pensamiento computacional desde educación básica, crear un centro nacional de IA y reformar la formación docente. El Ministerio de TIC ha operado programas como ``Programación para Niños y Niñas'' y la Ruta STEM. El Centro para la Cuarta Revolución Industrial (C4IR Colombia, establecido en Medellín en 2019 como parte de la red del Foro Económico Mundial) ha servido como laboratorio de políticas de IA, incluidos proyectos piloto en educación. El contenido de IA no se ha integrado formalmente en los Estándares Básicos de Competencias del currículo nacional, y la formación docente en IA se limita a programas piloto.


\section{Asia-Pacífico}
\label{sec:asia-pacifico}

\subsection{China}

China publicó el Plan de Desarrollo de IA de Nueva Generación en julio de 2017, uno de los primeros documentos nacionales en vincular explícitamente la IA con la reforma educativa~\citep{china2017ai}. El plan establece tres fases: para 2020, mantener paridad con la tecnología global; para 2025, lograr avances mayores; para 2030, convertirse en el centro mundial de innovación en IA. La educación es una prioridad estratégica explícita, con mandatos para establecer cursos de IA en primaria y secundaria, desarrollar libros de texto y construir bases de formación en IA.

El Ministerio de Educación emitió el Plan de Acción para la Innovación en IA en Instituciones de Educación Superior en 2018. En 2019, China lanzó un programa piloto que introdujo cursos de IA en más de 20,000 escuelas primarias y secundarias, apoyado por la primera serie de libros de texto aprobados por el gobierno, \textit{Fundamentos de Inteligencia Artificial}~\citep{pedro2019aieducation}. Para 2020, más de 440 universidades habían recibido aprobación para ofrecer licenciaturas en IA. El Plan de Informatización Educativa 2.0 (2018) promueve plataformas de aprendizaje adaptativo y campus inteligentes. La escala de implementación es la mayor del mundo por número de escuelas, pero enfrenta desafíos documentados: acceso desigual entre zonas urbanas y rurales, profundidad pedagógica variable y limitada integración de ética de IA en el currículo K-12~\citep{holmes2022state}.

\subsection{Japón}

Japón enmarca su estrategia de IA dentro de la visión Sociedad 5.0, articulada en el V Plan Básico de Ciencia y Tecnología (2016), que concibe una ``sociedad superinteligente'' que integra los espacios cibernético y físico. La AI Strategy 2019 estableció metas educativas concretas: todos los graduados de secundaria debían haber recibido alfabetización básica en IA, 250,000 estudiantes universitarios por año debían completar cursos de IA, y 2,000 estudiantes por año debían recibir formación especializada~\citep{japan2019ai}.

El GIGA School Programme (2019) invirtió aproximadamente 231,800 millones de yenes (unos 2,200 millones de dólares) para proporcionar un dispositivo por estudiante y acceso de banda ancha a cada escuela, acelerado por la pandemia. La programación se convirtió en contenido obligatorio en primaria desde abril de 2020, y la informática en asignatura obligatoria en secundaria superior desde abril de 2022, con conceptos de IA integrados. En educación superior, el gobierno fijó la meta de que todos los estudiantes universitarios completen cursos básicos de matemáticas, ciencia de datos e IA. Para 2023, aproximadamente el 80\% de las universidades nacionales habían establecido programas de este tipo. Una brecha documentada es la escasez de docentes certificados en informática: aproximadamente el 20\% de las clases de ciencias de la información en secundaria eran impartidas por profesores sin certificación especializada~\citep{oecd2021digitaloutlook}.

\subsection{Corea del Sur}

Corea del Sur publicó su Estrategia Nacional de IA en diciembre de 2019, seguida del Plan de Educación en IA para Todos en noviembre de 2020~\citep{korea2019ai}. El Plan de Educación es una de las políticas K-12 de IA más específicas del mundo: establece que la educación en IA sería obligatoria en todos los niveles a partir de 2025, con un mínimo de 34 horas en primaria, 68 horas en secundaria básica y cursos electivos y obligatorios en secundaria superior. El gobierno anunció un programa para formar 5,000 docentes especializados en IA para 2025.

En 2023, Corea del Sur lanzó la iniciativa de Libros de Texto Digitales con IA: libros de texto adaptativos basados en IA reemplazarían los textos tradicionales en matemáticas, inglés e informática a partir de 2025, con expansión a otras asignaturas para 2028. La estrategia se articula con el Korean New Deal (2020), cuyo componente ``Digital New Deal'' asignó aproximadamente 58.2 billones de won (unos 49,000 millones de dólares) a la transformación digital, con la educación como sector prioritario. Corea se ubica consistentemente entre los tres primeros puestos del Índice de Oxford Insights. Preocupaciones expresadas por sindicatos docentes incluyen la preparación para la implementación y la carga de trabajo adicional, así como el riesgo de privilegiar las habilidades técnicas sobre la reflexión ética.

\subsection{Singapur}

Singapur publicó su Estrategia Nacional de IA (NAIS) en noviembre de 2019, actualizada como NAIS 2.0 en diciembre de 2023~\citep{singapore2019nais}. El programa AI Singapore (AISG), establecido en 2017 con 150 millones de dólares singapurenses (aproximadamente 110 millones de dólares), funciona como el programa nacional de desarrollo de capacidades en IA.

La educación en IA abarca todas las etapas de la vida. En K-12, el programa \textit{Code for Fun} (expandido para incluir elementos de IA desde 2020) expone a alumnos de primaria y secundaria al pensamiento computacional y la IA en 10 horas de currículo. El \textit{AI Student Outreach Programme} (2021) proporciona experiencia práctica en proyectos de IA a estudiantes de secundaria. En educación terciaria, la Universidad Nacional de Singapur requiere que todos los estudiantes completen un módulo de alfabetización en datos e IA. SkillsFuture, el marco nacional de aprendizaje permanente, ofrece cursos y subsidios de IA para adultos. Singapur se ubicó en el primer lugar de Asia y segundo en el Índice de Oxford Insights 2023. Su escala reducida (5.9 millones de habitantes) facilita la coordinación pero limita la generalización de su modelo a países más grandes.

\subsection{India}

India publicó la Estrategia Nacional para la IA (\#AIForAll) en 2018 a través de NITI Aayog~\citep{india2018niti}. La Política Nacional de Educación (NEP) 2020, la primera revisión educativa importante en 34 años, aborda la IA explícitamente: propone integrar el pensamiento computacional, la programación y la IA desde la etapa intermedia del currículo, y establece un Foro Nacional de Tecnología Educativa (NETF)~\citep{india2020nep}.

El Central Board of Secondary Education (CBSE) introdujo un currículo electivo de IA para los grados IX y X desde el año escolar 2019--2020, lo que convierte a India en uno de los primeros países en ofrecer la IA como asignatura escolar formal. El programa \textit{Responsible AI for Youth}, lanzado por NITI Aayog en colaboración con Intel India, ha alcanzado a más de 500,000 estudiantes desde 2020. El marco \#AIForAll posiciona la IA como herramienta de desarrollo inclusivo, un enfoque de equidad distintivo. Sin embargo, la implementación es extremadamente desigual: el CBSE cubre solo una fracción de los estudiantes indios, los consejos estatales han sido lentos en adoptar el currículo de IA, aproximadamente el 30\% de las escuelas contaban con laboratorios de cómputo funcionales, y la formación docente en IA es mínima fuera de las grandes ciudades~\citep{southgate2020aiethics}.

\subsection{Australia}

Australia publicó su Marco de Ética de IA en 2019 y su Plan de Acción de IA en 2021, con 124.1 millones de dólares australianos de financiamiento~\citep{australia2021aiaction}. El Centro Nacional de IA, establecido dentro del CSIRO Data61, coordina programas de sensibilización y proporciona herramientas de autoevaluación ética.

El Australian Curriculum incluye el área de aprendizaje \textit{Digital Technologies}, revisada en la versión 9.0 (2022), que incorpora representación de datos, algoritmos y conceptos relevantes para la IA desde los primeros años hasta el décimo grado, obligatoria hasta octavo grado. En 2023, el gobierno australiano publicó el \textit{Australian Framework for Generative AI in Schools}, una de las primeras guías gubernamentales del mundo dirigidas específicamente al uso de IA generativa en escuelas~\citep{southgate2020aiethics}. En educación superior, el Plan de Acción financió 450 becas de IA y apoyó la creación de cuatro centros nacionales de excelencia en IA. Las brechas incluyen la ausencia de una estrategia dedicada de IA en educación (las provisiones están fragmentadas entre la autoridad curricular, el departamento de innovación y los sistemas educativos estatales), el presupuesto modesto frente a los pares asiáticos, y las dificultades de acceso en escuelas rurales y remotas.


\section{Síntesis Comparativa Preliminar}
\label{sec:sintesis-preliminar}

La revisión del panorama global permite identificar cinco patrones que el análisis formal examinará con mayor rigor en el capítulo de resultados.

\textbf{Primero, la educación no es el foco principal de las estrategias nacionales de IA.} De las 22 unidades de análisis, solo Brasil ubica la educación como primer eje estratégico de su política de IA. En la mayoría de los documentos, la educación aparece como un componente dentro de pilares más amplios de ``talento'', ``capital humano'' o ``sociedad''. Las estrategias priorizan la competitividad económica, la investigación y la industria; la educación ocupa un rol instrumental , es decir, proveer la fuerza laboral que la industria de IA necesita, más que un rol formativo propio~\citep{holmes2022state}.

\textbf{Segundo, existe una brecha pronunciada entre la formulación de políticas y su implementación curricular.} Varios países han formulado estrategias ambiciosas cuyas provisiones educativas permanecen sin ejecutar. Chile propuso integrar la IA en el currículo K-12 en 2021 pero permanece en fase piloto. La EBIA de Brasil carece de asignaciones presupuestarias para sus metas educativas. India introdujo un currículo de IA a través del CBSE pero este llega solo a una fracción del estudiantado. La distancia entre la retórica de los documentos y la realidad de las aulas es un hallazgo transversal.

\textbf{Tercero, la formación docente es la brecha más consistente.} Ninguna de las 22 unidades de análisis reporta un sistema integral de formación docente en IA. Corea del Sur, con su meta de 5,000 docentes especializados para 2025, es el caso más ambicioso, pero incluso allí los sindicatos docentes han expresado preocupaciones sobre la preparación. En Japón, el 20\% de las clases de informática son impartidas por docentes sin certificación. En México, Brasil, India y Colombia, la formación docente en IA es prácticamente inexistente fuera de proyectos piloto. Esta brecha limita el alcance de cualquier política curricular, por ambiciosa que sea~\citep{miao2021guidance}.

\textbf{Cuarto, los enfoques curriculares divergen en una cuestión central: IA como asignatura, como herramienta o como tema transversal.} China, Corea del Sur e India han optado por introducir la IA como asignatura formal con horas curriculares definidas. Francia ha desplegado herramientas de IA en aulas sin necesariamente enseñar sobre IA. Finlandia y la Unión Europea privilegian un enfoque de competencias transversales que integra la IA en la alfabetización digital general. Singapur combina los tres enfoques según el nivel educativo. Esta divergencia refleja concepciones distintas sobre qué significa ``educar en IA'' y plantea una pregunta que los marcos teóricos de alfabetización en IA~\citep{long2020ailiteracy, ng2021ailiteracy} intentan responder pero que las políticas nacionales han respondido de formas dispares.

\textbf{Quinto, la estructura de gobierno condiciona la capacidad de implementación.} Los países con sistemas educativos centralizados (China, Corea del Sur, Singapur, Francia) han logrado implementaciones más rápidas y uniformes. Los países con sistemas federales o descentralizados (Estados Unidos, Canadá, Alemania, Australia, México, Brasil, India) enfrentan fragmentación: las estrategias nacionales establecen la dirección pero la implementación depende de autoridades subnacionales con capacidades y prioridades diversas. La Unión Europea enfrenta una versión supranacional de este mismo problema. Este patrón sugiere que la gobernanza educativa puede ser un factor más decisivo que el presupuesto en la velocidad de integración de la IA en los sistemas educativos.

Estos cinco patrones (la instrumentalización de la educación, la brecha implementación-formulación, el déficit docente, la divergencia curricular y el factor gobernanza) constituyen las hipótesis de trabajo que el análisis semántico y cualitativo del capítulo de resultados pondrá a prueba.
